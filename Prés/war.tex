\RequirePackage[l2tabu, orthodox]{nag}
\documentclass[english, french]{beamer}
\input{preamble/packages}
\input{preamble/redac}

%\setbeamertemplate{headline}[singleline]
\setbeamertemplate{footline}[title]

\title{Nathanson, \emph{Terrorism and the Ethics of War} : compte-rendu}
\subject{Éthique}
\keywords{terrorisme, philosophie morale}
\author{Olivier Cailloux}
\date{\formatdate{6}{5}{2025}}

\begin{document}
\begin{frame}[plain]
  \tikz[remember picture,overlay]{
    \path (current page.south) node[anchor=south, inner sep=0] {
      \includegraphics[width=\paperwidth, trim = {0 3cm 0 0}]{Rideau.jpg}
    };
  }
  \titlepage
\end{frame}
\addtocounter{framenumber}{-1}

\AtBeginSection{
  \begin{frame}
    \frametitle{\translate{Outline}}
    \tableofcontents[currentsection, hideallsubsections]
  \end{frame}
}

\begin{frame}
  \frametitle{\translate{Outline}}
  \tableofcontents[hideallsubsections, sectionstyle=shaded/show]
\end{frame}

\section{Introduction}
\subsection{Contexte}
\begin{frame}
  \frametitle{Présentation}
  \begin{itemize}
    \item \href{https://dailynous.com/2023/02/14/stephen-nathanson-1944-2023/}{Nathanson} (1944-2023) “was known for his work on economic justice, war and peace, patriotism, punishment, and social and political philosophy”
    \item Importance de définir le terrorisme précisément
    \item Ne pas inclure de condamnation morale dans la définition
    \item L’obtenir analytiquement
    \item Le terrorisme est \emph{toujours} injustifié
    \item Lier à l’éthique de la guerre
    \item Livre paru en 2010
    \item Motivation : 9/11
    \item Test de la théorie sur l’actualité !
  \end{itemize}
\end{frame}

\subsection{Note méthodologique}
\begin{frame}
  \frametitle{Note méthodologique}
  Ceci est un compte rendu !
\end{frame}

\begin{frame}
  \frametitle{Se taire ?}
  Peut-on commenter les positions de l’auteur ?
  \begin{block}{En cas de désaccord avec Nathanson}
    \begin{quote}
      L'apologie du terrorisme consiste à présenter ou commenter favorablement soit le terrorisme en général, soit des actes terroristes déjà commis
    \end{quote}
    \vspace{-3mm}
    \small
    \raggedleft{\href{https://www.service-public.fr/particuliers/vosdroits/F32512}{Service-Public.fr}}%Apologie du terrorisme - Provocation au terrorisme, 
  \end{block}
  \begin{block}{En cas d’accord avec Nathanson}
    Défendre une thèse légalement intouchable, est-ce de la propagande ou du débat ?
  \end{block}
  La conclusion pourrait-elle être l’abstention ?
\end{frame}

\subsection{Complexité du débat}
\begin{frame}
  \frametitle{C’est pourtant simple !}
  \begin{itemize}
    \item Tuer des innocents, c’est toujours mal
    \item Le terrorisme consiste au moins à tuer des innocents
    \item Donc, le terrorisme, c’est toujours mal
  \end{itemize}
\end{frame}

\begin{frame}
  \frametitle{Liens avec l’éthique de la guerre}
  \begin{itemize}
    \item La guerre est parfois légitime (position pacifiste radicale non considérée ici)
    \item La guerre implique de tuer parfois des innocents
    \item Au nom d’une juste cause ? D’un dommage collatéral inévitable ?
    \item Requiert une éthique de la guerre
    \item S’applique au terrorisme
  \end{itemize}
\end{frame}

\begin{frame}
  \frametitle{Exemple}
  \begin{itemize}
    \item Bombardement de GB sur Allemagne en 1940
    \item Situation critique pour les alliés
    \item Cible principalement civile pour démoralisation de l’ennemi
  \end{itemize}
  \begin{block}{Walzer, \emph{Just and Unjust Wars}}
    Acte considéré terroriste mais moralement non condamné
    \vspace{2mm}\mbox{}
    \begin{quote}
      Nazism was an ultimate threat to everything decent in our lives, an ideology and a practice of domination so murderous, so degrading … that the consequences of its final victory were literally beyond calculation, immeasurably awful … Here was a threat to human values so radical that its imminence would surely constitute a supreme emergency.
    \end{quote}
  % \vspace{-3mm}
  % \small
  % \raggedleft{}
  \end{block}
\end{frame}

\section{Définition du terrorisme}
\begin{frame}
  \frametitle{Méthode}
  \begin{itemize}
    \item Certaines définitions sont meilleures que d’autres
    \item Définition précise requise
    \item Permet de s’accorder sur la qualification des actes
    \item Le débat moral peut alors être distingué
    \item Indépendant des auteurs
  \end{itemize}
\end{frame}

\begin{frame}
  \frametitle{Définition}
  Terrorist acts:
  \begin{enumerate}
    \item are acts of serious, deliberate violence or credible threats of such acts;
    \item are committed in order to promote a political or social agenda;
    \item generally target limited numbers of people but aim to influence a larger
    group and/or the leaders who make decisions for the group;
    \item intentionally kill or injure innocent people or pose a threat of serious
    harm to them.
  \end{enumerate}
\end{frame}

\begin{frame}
  \frametitle{Défense}
  \begin{itemize}
    \item S’applique aux états : leurs actes doivent être condamnés pour les mêmes raisons
    \item Ne dépend pas des motifs : lutte contre les nazis compris
  \end{itemize}
\end{frame}

\section{Immoralité systématique du terrorisme}
\subsection{Théories politiques}
\begin{frame}
  \frametitle{Réalisme politique}
  \begin{itemize}
    \item Pas de moralité, défendre le pays uniquement
    \item Les dirigeants d’un pays sont responsables de sa défense et de leurs citoyens uniquement
    \item Entraine immoralité extrême
    \item Viole les intuitions morales qui permettent d’être horrifié par le 11/9. On ne pourrait le condamner qu’en adoptant le pdv des É-U et en reconnaissant à Al-Quaida une action intelligente.
  \end{itemize}
\end{frame}

\begin{frame}
  \frametitle{Moralité du sens commun}
  \begin{itemize}
    \item Henry Stimson, secrétaire à la guerre sous Roosevelt et Truman
    \item Soutenait l’usage de l’arme atomique
  \end{itemize}
  \emph{The Decision to Use the Atomic Bomb}, 1947
  \begin{quote}
    My chief purpose was to end the war in victory with the least possible cost in the lives of the men in the armies which I had helped to raise … I believe that no man, in our position and subject to our responsibilities, holding in his hands a weapon of such possibilities for accomplishing this purpose and saving those lives, could have failed to use it and afterwards looked his countrymen in the face.
  \end{quote}
\end{frame}

\begin{frame}
  \frametitle{Problème avec ces théories morales}
  \begin{itemize}
    \item Très imprécises
    \item Ne permettent pas de trancher, et exposition aux risques des imprécisions (à venir !)
    \item Requiert d’argumenter du point de vue de son camp, pas de condamner les actes terroristes en soi
  \end{itemize}
\end{frame}

\begin{frame}
  \frametitle{Utilitarisme de l’acte}
  \begin{itemize}
    \item L’action juste maximise la somme des bien-êtres moins la somme des souffrances
    \item Faut-il considérer les soldats différemment des civils ?
    \item Possible d’accepter une règle de non terrorisme ?
    \item Mais même dans ce cas, refuse-t-on le terrorisme pour une raison solide ?
    \item Francisco de Vitoria : sack a city, permissible if it acted “as a spur to the courage of the troops” (critiqué par Walzer)
  \end{itemize}
  \begin{quote}
    It is particularly important not to lose confidence in our absolutist intuitions [including the belief that killing innocent people is wrong] for they are often the only barrier before the abyss of utilitarian apologetics for large-scale murder
  \end{quote}
  Thomas Nagel, \emph{War and Massacre}, 1972
\end{frame}

\begin{frame}
  \frametitle{Les droits humains}
  \begin{itemize}
    \item Les humains ont des droits
    \item Ils peuvent y renoncer (soldats)
    \item Dworkin accepte que les droits sont des trumps mais pas absolus
    \item Les droits, même à la vie, contestent d’autres droits : défense d’aggressions graves mais non mortelles ; bombe atomique évitant le combat à Fussell ; tirer pour prévenir une agression
  \end{itemize}
\end{frame}

\begin{frame}
  \frametitle{Exceptions ?}
  \begin{itemize}
    \item Walzer condamne, mais avec exceptions
    \item Exception au début de la guerre mais pas à la fin
    \item Objection : où s’arrêtent les supreme emergencies ?
  \end{itemize}
  \begin{quote}
    Walzer’s relatively benign view of Japanese aggression is hard to take seriously. I feel inclined to say “Tell that to the Chinese.” In the Japanese invasion of China in the 1930s it is soberly estimated that more than 300,000 Chinese civilians were massacred in Nanking alone in a racist rampage of raping and beheading and bayoneting that lasted six weeks. Nor was the racist and anti-civilisational behaviour of the Japanese warriors much better in the rest of South-East Asia during the war.
  \end{quote}
  Tony Coady, \emph{Terrorism, Just War and Supreme Emergency}, 2002
\end{frame}

\begin{frame}
  \frametitle{La guerre juste}
  \begin{itemize}
    \item Requiert un gain militaire substantiel et pas d’intention de nuire aux civils
    \item Dommage collatéraux acceptés
    \item Mais une version alternative du 11 septembre satisfairait cette définition mais serait à rejeter moralement néanmoins
    \item On ne peut pas détruire tout un quartier pour protéger quelques soldats
  \end{itemize}
\end{frame}

\subsection{Contraintes}
\begin{frame}
  \frametitle{Contraintes}
  \begin{itemize}
    \item Les dommages collatéraux civils sont acceptables seulement s’il y a proportionnalité et précaution
    \item Proportionnalité : le gain militaire est significatif et prouvable au regard des dommages vraisemblable aux civils
    \item Précaution : les auteurs ont fait preuve de tous les efforts raisonnables pour diminuer les dommages aux civils (y compris en mettant ses soldats en danger)
    \item Restrictions fortes qui interdisent les situations extrêmes
  \end{itemize}
\end{frame}

\begin{frame}
  \frametitle{Et le point de vue des opprimés ?}
  Objection de Primoratz à Coady (\emph{Civilian Immunity in War})
  \begin{quote}
    Think of a people facing the prospect of genocide, or of being “ethnically cleansed” from its land, and unable to put up a fight against an overwhelmingly stronger enemy. Suppose we said to them: “Granted, what you are facing is an imminent threat of a moral disaster. Granted, the only way you stand a chance of fending off the disaster is by acting in breach of the principle of civilian immunity and attacking enemy civilians. But you must not do that. (…)” Could they – indeed, should they – be swayed by that?
  \end{quote}
  \vspace{-5mm}
  \begin{itemize}
    \item Il se pourrait que ça ne convainque pas en effet
    \item Mais pas normatif
    \item Un point de vue neutre a priori doit considérer les citoyens innocents
  \end{itemize}
\end{frame}

\begin{frame}
  \frametitle{Arguments pour ces restrictions}
  \begin{itemize}
    \item Défense par l’utilitarisme de la règle
    \item Autoriser les crimes pour raccourcir la guerre, pour éviter des catastrophes suprêmes ?
    \item Mais : distinguer l’intérêt et ce que les politiciens croient, ou font croire, être l’intérêt
    \item Ne pas encourager à l’escalade ou la banalisation de la violence
    \item S’interdire de remettre la règle en question au cas par cas
    \item Car risque de biais lors de l’évaluation de ses propres chances de succès ; du découragement de l’ennemi ; des dommages vraisemblables ; de l’urgence de la situation
  \end{itemize}
\end{frame}

\section{Conclusion}
\begin{frame}
  \frametitle{Conclusion}
  \begin{itemize}
    \item Progrès vers une clarification des concepts et termes
    \item Méthode proposée : Terrorisme factuel ; condamnation morale analytique sans considération de camp, d’urgence…
    \item Nombreuses positions morales ne condamnant pas totalement le terrorisme
    \item Ici : pas de terrorisme justifié
    \item Défense de l’utilitarisme de la règle
    \item Question terminologique : appeler ça du terrorisme ?
  \end{itemize}
\end{frame}

\begin{frame}[plain]
  \addtocounter{framenumber}{-1}
  \begin{center}
    \huge
    \textit{Thank you for your attention!}
  \end{center}
\end{frame}

\end{document}

\appendix
\AtBeginSection{
}

\begin{frame}[allowframebreaks]
  \frametitle{\refname}
   \bibliography{zotero}
\end{frame}

\clearpage\pdfbookmark{License}{License}
\begin{frame}[plain]
  \frametitle{License}
  This presentation, and the associated \LaTeX{} code, are published under the \href{https://opensource.org/licenses/MIT}{MIT license}. Feel free to reuse (parts of) the presentation, under condition that you cite the author.
  
  Credits are to be given to \hrefblue{https://www.lamsade.dauphine.fr/~ocailloux/}{Olivier Cailloux}, Université Paris-Dauphine.
\end{frame}
\addtocounter{framenumber}{-1}
\end{document}

\begin{frame}
  \frametitle{Title}
  \begin{itemize}
    \item Item
  \end{itemize}
\end{frame}

\begin{frame}
  \frametitle{Title}
  \begin{block}{Block}
%		\setlength\abovedisplayskip{1 ex}% reduce space above equations
    \begin{itemize}
      \item Item
    \end{itemize}
  \end{block}
  \begin{itemize}
    \item Item
  \end{itemize}
\end{frame}

